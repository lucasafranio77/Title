\chapter{Introdução}
A Introdução anuncia o que se pretende dizer. Deve resumir o tema-problema a ser abordado, sua contextualização, o raciocínio a ser seguido na solução e os métodos a serem adotados, deixando claros o objetivo do trabalho, sua relevância, a justificativa da abordagem do seu tema, as propostas formuladas e a composição do trabalho. O leitor
deve ter uma ideia geral clara a respeito do que vai ler, após percorrer a introdução. 

É onde se faz a apresentação do problema, ressaltando os motivos mais importantes que levaram à abordagem do tema. O tema ou problema de pesquisa deve responder a o que vai ser investigado. A escolha do tema não é criação individual do aluno, mas é feita com base em obras que o abordem, trabalhadas por outros autores. A perspectiva adotada deve, contudo, ser diferente, a partir da consulta à documentação para a realização do projeto. O tema escolhido deve ser tratado como um problema a ser resolvido. O desenvolvimento do trabalho visa demonstrar uma posição única a respeito do tema problematizado. Trata-se de definir os aspectos de dificuldade e de se esclarecer os limites dentro dos quais a pesquisa e o raciocínio se desenvolverão.

Para decidir sobre o tema, algumas questões precisam ser respondidas, tais como:

\begin{itemize}
\item O tema é de interesse científico?
\item É um assunto a ser provado ou resolvido?
\item É um assunto que pode ser investigado?
\item Há a disponibilidade de material bibliográfico sobre o assunto?
\item Estou familiarizado com o tema?
\item A pesquisa é viável, em termos de tempo e recursos disponíveis?
\end{itemize}

O problema, por sua vez, é manifestação da vontade de investigar o tema, consistindo-se em uma questão não resolvida.

Exemplo: A adoção da técnica XYZ de avaliação pode corrigir os problemas com a evolução no aprendizado do aluno?

A Introdução deve conter:

1.1 Objetivo Geral

1.2 Objetivos Específicos

1.3 Justificativa

1.4 Metodologia

1.5 Cronograma Proposto

\section{Objetivo Geral}

O aluno expõe os objetivos que o trabalho visa atingir, relacionados com a contribuição
que pretende trazer. Os objetivos visam responder para que o trabalho será realizado.
Os objetivos devem ser formulados com verbos no infinitivo.

O \textbf{Objetivo Geral} expõe a razão maior do trabalho, ou o que se pretende com a realização do trabalho. A formulação do \textbf{objetivo gera}l usa verbos que admitem interpretações amplas, como \textbf{comprovar}, \textbf{desenvolver},
\textbf{entender}, \textbf{conhecer} e
\textbf{aperfeiçoar}.

Exemplo: O objetivo geral deste trabalho é estabelecer processos de apoio ao desenvolvimento das aplicações Web, adaptados à natureza dessas aplicações e tomando como base os aspectos observados na literatura sobre as características
dessas aplicações.

\section{Objetivos Específicos}
Os Objetivos Específicos relacionam os resultados que se pretende alcançar, na forma de etapas a serem cumpridas para a realização do objetivo geral. Os objetivos específicos são o ponto de partida para a investigação, ordenados
em uma sequência lógica de obtenção dos resultados desejados.
Os verbos para formular objetivos específicos devem admitir poucas interpretações, tais como \textbf{identificar}, \textbf{implementar}, \textbf{investigar}, \textbf{relacionar}, \textbf{escrever} e \textbf{aplicar}.

Exemplo:
\begin{itemize}
 \item Pesquisar sobre a evolução das arquiteturas de sistemas Web;
 \item Estudar as diferentes visões arquiteturais;
 \item Definir critérios para comparar as arquiteturas estudadas;
\end{itemize}

\section{Justificativa}

As justificativas devem ser baseadas na relevância social e científica da pesquisa proposta. Nesta etapa é respondido \textbf{porque} o trabalho será feito. A justificativa deve deixar claro para o leitor:

\begin{itemize}
 \item O estágio de desenvolvimento em que o tema se encontra e a sua evolução histórica;
\item O contexto em que o fenômeno ocorre;
\item A importância social e científica da realização da pesquisa sobre o tema.
\end{itemize}


\section{Metodologia}

Aqui se anuncia, para o projeto, o tipo de pesquisa que será desenvolvida, bem como os métodos e técnicas que serão adotados. Este item visa responder às questões sobre \textbf{como} o trabalho será feito,\textbf{ com o que},
\textbf{onde} e \textbf{com quem} será realizado.
Exemplo: O trabalho será desenvolvido a partir de pesquisa bibliográfica e de campo, através de método indutivo, utilizando levantamento de dados feito em entrevistas com professores e coordenadores acadêmicos, $\ldots$



\section{Cronograma}

A divisão da pesquisa em etapas de desenvolvimento, a partir dos objetivos específicos,requer que seja estabelecido um tempo para a execução cronológica dos trabalhos, dentro dos prazos estabelecidos para tal no calendário acadêmico. A elaboração do
cronograma responde à pergunta sobre \textbf{quando} cada fase do trabalho terá lugar. As etapas do cronograma devem cobrir todo o período de realização do trabalho de monografia, desde o refinamento do Projeto de Monografia até a apresentação do
trabalho monográfico à banca de avaliação. Cada etapa a ser cumprida deve ser exibida como uma linha na tabela que compõe o cronograma. As etapas são descritas após a exibição da tabela, tal qual mostra a tabela \ref{tab:Cronograma}.

\textbf{Exemplo}:

\begin{table}[h]
\caption{Cronograma Proposto}
\begin{tabular}{|c|c|c|c|c|c|c|c|c|c|c|c|c|c|c|c|c|}
\hline
& \multicolumn{16}{c|}{\textbf{Meses/Semanas}} \\
\hline
\multirow{2}{*}{\textbf{Etapas}} & \multicolumn{4}{c|}{\textbf{Mês 1}} & \multicolumn{4}{c|}{\textbf{Mês 2}} & \multicolumn{4}{c|}{\ldots} & \multicolumn{4}{c|}{\textbf{Mês N}}  \\
\cline{2-17}
        & 1 & 2 & 3 & 4 & 1 & 2 & 3 & 4 & 1 & 2 & 3 & 4 & 1 & 2 & 3 & 4 \\
\hline        
Etapa 1 & X & X & X & X &   &   &   &   &   &   &   &   &   &   &   &  \\
\hline 
Etapa 2 &   &   &   & X & X & X &   &   &   &   &   &   &   &   &   &  \\ 
\hline
Etapa 3 &   &   &   &   &   &   & X & X & X &   &   &   &   &   &   &  \\
\hline 
\ldots  &   &   &   &   &   &   &   &   &   & X & X & X & X & X & X &  \\ 
\hline
Etapa N &   &   &   &   &   &   &   &   &   &   &   &   &   &   &   & X \\
\hline 
\end{tabular}
\label{tab:Cronograma}
\end{table}
\begin{description}
\item [\textbf{Etapa 1}] - Exemplo: Pesquisa bibliográfica
Consiste da leitura de livros sobre o tema...
\item [\textbf{Etapa 2}] - Exemplo: Estudo de Caso
\ldots
\item[\textbf{Etapa N}] - Exemplo: Apresentação à banca avaliadora
\end{description}


\section{Estudo de Caso}
...
