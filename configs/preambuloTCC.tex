% Versão Março de 2016
% Autor: Mauricio F. L Pereira
% e-mail: {mauricio, andreia.bonfante}@ic.ufmt.br
% 
%% MODELO DE MONOGRAFIA DO INSTITUTO DE COMPUTAÇÃO - UFMT


%% Abaixo são exemplos para cada tipo de trabalho 
%  Deixe apenas um dos \documentclass[]{} ativos, de acordo 
% com a fase (PTCC ou TCC ) e categoria (Graduação ou Especialização) 
% do seu trabalho de conclusão.
% Neste exemplo está ativado o exemplo de PTCC de Especialização em Banco de Dados
%% PTCC  (Especializacao, Banco de Dados)
%\documentclass[EspecializacaoPTCC,BD]{./configs/icufmt}

%% TCC  (Especializacao, Banco de Dados)
%\documentclass[EspecializacaoTCC,BD]{./configs/icufmt}

%% PTCC (Especializacao, Eng Web)
% \documentclass[EspecializacaoPTCC,EngWeb]{./configs/icufmt}

%% TCC (Especializacao, Eng Web)
%\documentclass[EspecializacaoTCC,EngWeb]{./configs/icufmt}

%% PTCC ( Graduacao Ciencia da Computação)
\documentclass[PTCC]{./configs/icufmt}

%% TCC ( Graduacao Ciencia da Computação)
%% \documentclass[TCC]{configs/icufmt}

\usepackage[brazil]{babel}

% para pacotes matemáticos
\usepackage{amsmath}
\usepackage{amsfonts}  
\usepackage{amssymb}  

%pacote para inclusão de códigos fonte
\usepackage{listings}

%pacote para lidar com tabelas com multiplas colunas
\usepackage{multirow}
	
% ---
% Pacotes adicionais, usados apenas para exemplos
% ---
\usepackage{lipsum}	% para geração de dummy text (pode ser comentado fora deste exemplo)
\usepackage{listings} % inclusão de códigos-fonte de diversas linguagens