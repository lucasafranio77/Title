\chapter{Exemplo de códigos-fonte}

Neste capítulo é apresentado como inserir códigos-fontes no trabalho. Para isso é necessário a utilização do pacote listings, que foi adicionado no arquivo principal da monografia.

Exemplo de código java


\lstset{language=java}
\lstinputlisting{exemplo.java}


\pagebreak
Outro exemplo de como inserir código com caixa para caption

\lstset{
language = C, % Linguagem de programação
basicstyle = \footnotesize, % Tamanho da fonte do código
numbers = left, % Posição da numeração das linhas
numberstyle = \tiny\color{blue}, % Estilo da numeração de linhas
stepnumber = 1, % Numeração das linhas ocorre a cada quantas linhas?
numbersep = 10pt, % Distância entre a numeração das linhas e o código
backgroundcolor = \color{white}, % Cor de fundo
showspaces = false, % Exibe espaços com um sublinhado
showstringspaces = false, % Sublinha espaços em Strings
showtabs = false, % Exibe tabulação com um sublinhado
frame = single, % Envolve o código com uma moldura, pode ser single ou trBL
rulecolor = \color{black}, % Cor da moldura
tabsize = 2, % Configura tabulação em x espaços
captionpos = b, % Posição do título pode ser t (top) ou b (bottom)
breaklines = true, % Configura quebra de linha automática
breakatwhitespace= false, % Configura quebra de linha
title = \lstname, % Exibe o nome do arquivo incluido
%caption = \lstname, % Também é possível usar caption no lugar de title
keywordstyle = \color{blue}, % Estilo das palavras chaves
commentstyle = \color{green}, % Estilo dos Comentários
stringstyle = \color{red}, % Estilo de Strings
escapeinside = {\%*}{*)}, % Permite adicionar comandos LaTeX dentro do seu código
morekeywords     ={*,...} % Se quiser adicionar mais palavras-chave
}
\begin{lstlisting}{Exemplo de Caption de Código}
#include <stdio.h>

int main()
{
// declaracao de variavel
int c;
printf("Entre com o valor: ");
scanf("%i", &c);
printf("Res: %i", i*i);
}
\end{lstlisting}