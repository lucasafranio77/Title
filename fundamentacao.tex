\chapter{Fundamentação Teórica}
\thispagestyle{empty}
Aqui se forma um quadro teórico de princípios e conceitos, dentro do qual o Projeto de Monografia se baseia e se desenvolve. Este quadro teórico deve formar uma unidade lógica compatível com o tratamento do problema e com o raciocínio desenvolvido. Ele nem sempre precisa constar do trabalho final, com o mesmo grau de profundidade adotado para o Projeto de Monografia, uma vez que seja mantida a coerência entre o desenvolvimento realizado e o projeto.

É onde se desenvolve a ideia anunciada, através da discussão dos elementos teóricos e empíricos. A Fundamentação Teórica pode ser composta por subcapítulos, tópicos e sub-tópicos, identificados por títulos que forneçam a ideia exata do seu conteúdo. Nesta parte do trabalho as ideias são descritas, classificadas e definidas, as ideias conflitantes são comparadas, e a argumentação apropriada à natureza do trabalho é aplicada.

É importante observar que os conteúdos apresentados por outros autores devem ser referenciados no texto de acordo com as normas ABNT NBR 6023 e NBR 10520 